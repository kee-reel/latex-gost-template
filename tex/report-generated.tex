Вычислить значение функции:\\
\begin{equation*}f(x,y) =\begin{cases}a * u, & \textrm{если a <= u, u < 0,}\\u^2 - a, & \textrm{если u > 0,}\\u & \textrm{в остальных случаях.}\end{cases},\end{equation*}\\
При вычислении значения необходимо использовать условную операцию сравнения ?:



\chapter*{Задание 1. Функция}

Вычислить значение функции:\\\n\begin{equation*}f(x,y) =\begin{cases}x + y, & \textrm{если x > 0,}\\x * y, & \textrm{если x <= 0, y < 0,}\\5 * x & \textrm{в остальных случаях.}\end{cases},\end{equation*}
При вычислении значения необходимо использовать условную операцию сравнения ?:

\paragraph{Входные параметры}

\begin{tabular}{ |c|c|c|c| }
\hline
Номер параметра & Название & Тип & Допустимые значения \\ 
 \hline
1 & x & int & [-100, 100] \\ 
 \hline
2 & y & int & [-100, 100] \\ 
 \hline

\end{tabular}


\paragraph{Выходные параметры}

Вывести на экран целое значение функции.

\paragraph{Таблица тестирования}

\begin{tabular}{ |c|c|c| }
\hline
x & y & Результат\\
\hline
0 & 0 & 0 \\\n \hline
0 & 1 & 0 \\\n \hline
1 & 0 & 1 \\\n \hline
1 & 1 & 2 \\\n \hline
-1 & 0 & -5 \\\n \hline
0 & -1 & 0 \\\n \hline
-1 & -1 & 1 \\\n \hline
1 & -1 & 0 \\\n \hline
-1 & 1 & -5 \\\n \hline
-25 & 85 & -125 \\\n \hline
-41 & 36 & -205 \\\n \hline
35 & -52 & -17 \\\n \hline
-77 & 85 & -385 \\\n \hline
64 & -27 & 37 \\\n \hline
42 & 83 & 125 \\\n \hline
\end{tabular}


\paragraph{Листинг}
\begin{lstlisting}
#include <stdio.h>

#include <test_utils.h>

int main(int argc, char* argv[])
{
	ARG(1, int, x);
	ARG(2, int, y);
	printf("%d", x > 0 ? x + y : (x <= 0 && y < 0 ? x * y : 5 * x));
	return 0;
}

\end{lstlisting}
\\
\chapter*{Задание 2. Функция}

Вычислить значение функции:\\\n\begin{equation*}z = x^2 - \cos{\frac{\ln{ \sqrt{|x|} } }{\tg{e^{-x}}} }\end{equation*}
Необходимо определить ООФ и добавить проверки на входные параметры.
Для вычисления значений функций воспользуйтесь математической библиотекой math.h

\paragraph{Входные параметры}

\begin{tabular}{ |c|c|c|c| }
\hline
Номер параметра & Название & Тип & Допустимые значения \\ 
 \hline
1 & x & double & [-100, 100] \\ 
 \hline

\end{tabular}


\paragraph{Выходные параметры}

Вывести на экран вещественное значение функции с точностью до 2-х знаков.
Если значение выходит за ООФ, выводить "Не определено".

\paragraph{Таблица тестирования}

\begin{tabular}{ |c|c| }
\hline
x & Результат\\
\hline
-1.0 & 0.00 \\\n \hline
1.0 & 0.00 \\\n \hline
0.0 & -nan \\\n \hline
0.00001 & -0.29 \\\n \hline
-0.00001 & -0.29 \\\n \hline
1.1111 & 0.25 \\\n \hline
1000 & -nan \\\n \hline
-1000 & 1000000.95 \\\n \hline
-88.605532 & 7851.56 \\\n \hline
-86.218128 & 7434.18 \\\n \hline
-17.706825 & 313.40 \\\n \hline
-3.214558 & 9.50 \\\n \hline
-15.820694 & 250.11 \\\n \hline
\end{tabular}


\paragraph{Листинг}
\begin{lstlisting}
#include <math.h>
#include <stdio.h>
#define E_NUM 2.71828

#include "test_utils.h"

int main(int argc, char* argv[])
{
	ARG(1, double, x);
	printf("%.2lf", pow(x, 2) - cos(log(sqrt(fabs(x))) / tanh(pow(E_NUM, -x)) ));
	return 0;
}

\end{lstlisting}
\\
\chapter*{Задание 3. Координатная плоскость}

Вывести на экран номер четверти, которой принадлежит точка с координатами (x,y), или указать, какой оси принадлежит эта точка.
Необходимо использовать вложенный условный оператор.

\paragraph{Входные параметры}

\begin{tabular}{ |c|c|c|c| }
\hline
Номер параметра & Название & Тип & Допустимые значения \\ 
 \hline
1 & x & int & [-100, 100] \\ 
 \hline
2 & y & int & [-100, 100] \\ 
 \hline

\end{tabular}


\paragraph{Выходные параметры}

Вывести целое значение, обозначающее номер четверти: 1 -- правая верхняя, 2 -- левая верхняя, 3 -- левая нижняя, 4 -- правая нижняя. Если точка лежит на одной из координатных осей, то необходимо вывести 0.

\paragraph{Таблица тестирования}

\begin{tabular}{ |c|c|c| }
\hline
x & y & Результат\\
\hline
0 & 0 & 0 \\\n \hline
0 & 1 & 0 \\\n \hline
1 & 0 & 0 \\\n \hline
1 & 1 & 1 \\\n \hline
0 & -1 & 0 \\\n \hline
-1 & 0 & 0 \\\n \hline
-1 & -1 & 3 \\\n \hline
1 & -1 & 0 \\\n \hline
-1 & 1 & 2 \\\n \hline
-44 & 31 & 2 \\\n \hline
-70 & 24 & 2 \\\n \hline
-95 & 90 & 2 \\\n \hline
-52 & -26 & 3 \\\n \hline
-83 & -14 & 3 \\\n \hline
-100 & 9 & 2 \\\n \hline
\end{tabular}


\paragraph{Листинг}
\begin{lstlisting}
#include <stdio.h>

#include "test_utils.h"

int main(int argc, char* argv[])
{
	ARG(1, int, x);
	ARG(2, int, y);
	int res = 0;
	if(y > 0)
	{
	if(x > 0)
	{
		res = 1;
	}
	else if(x < 0)
	{
		res = 2;
	}
	}
	else if(y < 0)
	{
	if(x < 0)
	{
		res = 3;
	}
	else if(x < 0)
	{
		res = 4;
	}
	}
	printf("%d", res);
	return 0;
}

\end{lstlisting}
\\
\chapter*{Задание 4. Последовательность}

На вход подаётся натуральное число N. Вывести на экран последовательность вида "S=2*4*6*...*N", если N четное, или "S=1*3*5*...*N", если N нечетное.
Необходимо использовать цикл for.

\paragraph{Входные параметры}

\begin{tabular}{ |c|c|c|c| }
\hline
Номер параметра & Название & Тип & Допустимые значения \\ 
 \hline
1 & N & int & [2, 30] \\ 
 \hline

\end{tabular}


\paragraph{Выходные параметры}

Вывести на экран последовательность чисел. Например, при N = 7 вывод должен быть таким: "S=1*3*5*7".

\paragraph{Таблица тестирования}

\begin{tabular}{ |c|c| }
\hline
N & Результат\\
\hline
2 & S=2 \\\n \hline
3 & S=1*3 \\\n \hline
4 & S=2*4 \\\n \hline
5 & S=1*3*5 \\\n \hline
15 & S=1*3*5*7*9*11*13*15 \\\n \hline
20 & S=2*4*6*8*10*12*14*16*18*20 \\\n \hline
15 & S=1*3*5*7*9*11*13*15 \\\n \hline
5 & S=1*3*5 \\\n \hline
3 & S=1*3 \\\n \hline
29 & S=1*3*5*7*9*11*13*15*17*19*21*23*25*27*29 \\\n \hline
20 & S=2*4*6*8*10*12*14*16*18*20 \\\n \hline
\end{tabular}


\paragraph{Листинг}
\begin{lstlisting}
#include <stdio.h>

#include "test_utils.h"

int main(int argc, char* argv[])
{
	ARG(1, int, N);
	printf("S=");
	for(int i = 2 - N % 2; i <= N; i+=2)
	{
	printf("%d", i);
	if(i != N)
		printf("*");
	}
	return 0;
}

\end{lstlisting}
\\
\chapter*{Задание 5. Число Фиббоначи}

Вывести первое найденное число Фибоначчи, большее заданного n. Каждый член последовательности Фибоначчи является суммой двух предыдущих: xn = xn-1 + xn-2, x0 = 0, x1 = 1.
Необходимо использовать цикл while.

\paragraph{Входные параметры}

\begin{tabular}{ |c|c|c|c| }
\hline
Номер параметра & Название & Тип & Допустимые значения \\ 
 \hline
1 & n & int & [1, 1000] \\ 
 \hline

\end{tabular}


\paragraph{Выходные параметры}

Вывести на экран найденное число.

\paragraph{Таблица тестирования}

\begin{tabular}{ |c|c| }
\hline
n & Результат\\
\hline
1 & 2 \\\n \hline
5 & 8 \\\n \hline
1000 & 1597 \\\n \hline
31 & 34 \\\n \hline
192 & 233 \\\n \hline
138 & 144 \\\n \hline
910 & 987 \\\n \hline
211 & 233 \\\n \hline
133 & 144 \\\n \hline
\end{tabular}


\paragraph{Листинг}
\begin{lstlisting}
#include <stdio.h>

#include "test_utils.h"

int main(int argc, char* argv[])
{
	ARG(1, int, n);
	int a = 0, b = 1, temp;
	while(b <= n)
	{
	temp = b;
	b += a;
	a = temp;
	}
	printf("%d", b);
	return 0;
}

\end{lstlisting}
\\
\chapter*{Задание 6. Сократить дробь}

M и N – числитель и знаменатель обыкновенной дроби. Составить программу, позволяющую сократить эту дробь.
Необходимо использовать цикл do ... while.

\paragraph{Входные параметры}

\begin{tabular}{ |c|c|c|c| }
\hline
Номер параметра & Название & Тип & Допустимые значения \\ 
 \hline
1 & M & int & [-100, 100] \\ 
 \hline
2 & N & int & [-100, 100] \\ 
 \hline

\end{tabular}


\paragraph{Выходные параметры}

Вывести сокращённую дробь на экран в формате "целая-часть числитель/знаменатель" -- например: "-18/8" -> "-2 1/4".

\paragraph{Таблица тестирования}

\begin{tabular}{ |c|c|c| }
\hline
M & N & Результат\\
\hline
0 & 0 & - \\\n \hline
0 & 1 & 0 \\\n \hline
1 & 0 & - \\\n \hline
1 & 1 & 1 \\\n \hline
16 & 16 & 1 \\\n \hline
-16 & -4 & -4 \\\n \hline
-3 & -66 & -1/22 \\\n \hline
120 & 1 & 120 \\\n \hline
27 & 13 & 2 1/13 \\\n \hline
123 & -30 & -4 1/10 \\\n \hline
36 & -59 & -36/59 \\\n \hline
42 & 51 & 42/51 \\\n \hline
-34 & -67 & -34/67 \\\n \hline
-91 & 92 & -91/92 \\\n \hline
96 & -35 & -2 26/35 \\\n \hline
-48 & 4 & -12 \\\n \hline
\end{tabular}


\paragraph{Листинг}
\begin{lstlisting}
#include <stdio.h>

#include "test_utils.h"

int main(int argc, char* argv[])
{
	ARG(1, int, M);
	ARG(2, int, N);
	if(N == 0)
	{
		printf("-");
		return 0;
	}
	char is_negative = M < 0 || N < 0;
	M = M < 0 ? -M : M;
	N = N < 0 ? -N : N;

	int whole = M / N;
	M -= N * whole;
	while(M > 1 && N % M == 0)
	{
		N /= M;
		M /= M;
	}
	if(whole || !M)
	{
		if(is_negative)
			printf("-");
		printf("%d", whole);
	}
	if(M)
	{
		if(whole)
			printf(" ");
		if(is_negative && !whole)
			printf("-");
		printf("%d/%d", M, N);
	}
	return 0;
}

\end{lstlisting}
\\
\chapter*{Задание 7. Сумма бесконечного ряда}

Вывести на экран значение суммы бесконечного ряда
 \begin{equation}S = 1 - \frac{x}{1!} + \frac{x^2}{2!} + ... + (-1)^n*\frac{x^n}{n!}\end{equasion}
 С заданной точностью  \begin{equation}E = 10^{-4}\end{equasion}
 Сравнить полученное значение с целевым можно через функцию:  \begin{equation}f = e^-{x}\end{equasion}
Вычисление необходимо производить с помощью цикла, используя значения с предыдущих итераций.

\paragraph{Входные параметры}

\begin{tabular}{ |c|c|c|c| }
\hline
Номер параметра & Название & Тип & Допустимые значения \\ 
 \hline
1 & x & double & [-10, 10] \\ 
 \hline

\end{tabular}


\paragraph{Выходные параметры}

Вывести на экран вещественное число с точностью до указанного знака.

\paragraph{Таблица тестирования}

\begin{tabular}{ |c|c| }
\hline
x & Результат\\
\hline
-1.0 & 2.7183 \\\n \hline
1.0 & 0.3679 \\\n \hline
0.0 & 1.0000 \\\n \hline
0.5 & 0.6065 \\\n \hline
-0.5 & 1.6487 \\\n \hline
5.0 & 0.0067 \\\n \hline
-4.5 & 90.0171 \\\n \hline
-3.437609 & 31.1125 \\\n \hline
3.689325 & 0.0250 \\\n \hline
5.660770 & 0.0035 \\\n \hline
8.896451 & 0.0001 \\\n \hline
-1.615731 & 5.0316 \\\n \hline
\end{tabular}


\paragraph{Листинг}
\begin{lstlisting}
#include <math.h>
#include <stdio.h>
#define E_NUM 2.71828

#include "test_utils.h"

int main(int argc, char* argv[])
{
	ARG(1, double, x);
	double sum = 1;
	double x_mul = x;
	int fact = 1;
	char sign = -1;
	double a = 1;
	while(fabs(a) > 0.0001)
	{
	a = sign * x_mul;
	for(int i = 1; i <= fact; i++)
		a /= i;
	sum += a;
	x_mul *= x;
	fact++;
	sign *= -1;
	}
	printf("%.4lf", sum);
	return 0;
}

\end{lstlisting}
\\
\chapter*{Задание 8. Сумма бесконечного ряда}

Вывести на экран значение суммы бесконечного ряда \begin{equation}f(x)=\frac{x*\cos{\frac{\pi}{3}}}{1} + \frac{x^2*\cos{\frac{2*\pi}{3}}}{2} + ... + \frac{x^n*\cos{n*\frac{\pi}{3}}}{n} + ...\end{equasion}. С заданной точностью \begin{equation}E = 10^{-6}\end{equasion}.
Значение функции для проверки: \begin{equation}y=-\frac{1}{2} * \ln{1 - 2*x * \cos{\frac{\pi}{3}} + x^2}\end{equasion}
Вычисление необходимо производить с помощью цикла, используя значения с предыдущих итераций.

\paragraph{Входные параметры}

\begin{tabular}{ |c|c|c|c| }
\hline
Номер параметра & Название & Тип & Допустимые значения \\ 
 \hline
1 & x & double & [0.1, 0.8] \\ 
 \hline

\end{tabular}


\paragraph{Выходные параметры}

Вывести на экран вещественное число с точностью до указанного знака.

\paragraph{Таблица тестирования}

\begin{tabular}{ |c|c| }
\hline
x & Результат\\
\hline
0.1 & 0.047206 0.047206 \\\n \hline
0.8 & 0.087615 0.087616 \\\n \hline
0.100001 & 0.047206 0.047206 \\\n \hline
0.4 & 0.137460 0.137461 \\\n \hline
0.799999 & 0.087615 0.087616 \\\n \hline
0.491693 & 0.144096 0.144096 \\\n \hline
0.692686 & 0.120087 0.120086 \\\n \hline
0.456511 & 0.142861 0.142861 \\\n \hline
0.647146 & 0.129997 0.129995 \\\n \hline
0.198806 & 0.086858 0.086859 \\\n \hline
\end{tabular}


\paragraph{Листинг}
\begin{lstlisting}
#include <math.h>
#include <stdio.h>
#define PI_NUM 3.14

#include "test_utils.h"

int main(int argc, char* argv[])
{
	ARG(1, double, x);
	double sum = 0;
	double x_mul = x;
	int n = 1;
	double a;
	do
	{
	a = x_mul * cos(n * PI_NUM/3.) / n;
	sum += a;
	x_mul *= x;
	n++;
	}
	while(fabs(a) > 0.000001);
	printf("%lf ", -0.5 * log(1 - 2 * x * cos(PI_NUM/3) + x * x));
	printf("%.6lf", sum);
	return 0;
}

\end{lstlisting}
\\
\chapter*{Задание 9. Не простые числа}

Дано целое k. Вывести на экран все числа из диапазона [2,k], не являющиеся простыми. Простые числа не имеют других делителей, кроме 1 и самого себя.
Необходимо использовать вложенные циклы.

\paragraph{Входные параметры}

\begin{tabular}{ |c|c|c|c| }
\hline
Номер параметра & Название & Тип & Допустимые значения \\ 
 \hline
1 & k & int & [2, 100] \\ 
 \hline

\end{tabular}


\paragraph{Выходные параметры}

Вывод чисел в формате: "2,4,6,8,".

\paragraph{Таблица тестирования}

\begin{tabular}{ |c|c| }
\hline
k & Результат\\
\hline
2 & 2, \\\n \hline
5 & 2,3,5, \\\n \hline
47 & 2,3,5,7,11,13,17,19,23,29,31,37,41,43,47, \\\n \hline
71 & 2,3,5,7,11,13,17,19,23,29,31,37,41,43,47,53,59,61,67,71, \\\n \hline
3 & 2,3, \\\n \hline
48 & 2,3,5,7,11,13,17,19,23,29,31,37,41,43,47, \\\n \hline
4 & 2,3, \\\n \hline
94 & 2,3,5,7,11,13,17,19,23,29,31,37,41,43,47,53,59,61,67,71,73,79,83,89, \\\n \hline
4 & 2,3, \\\n \hline
\end{tabular}


\paragraph{Листинг}
\begin{lstlisting}
#include <math.h>
#include <stdio.h>

#include "test_utils.h"

int main(int argc, char* argv[])
{
	ARG(1, int, k);
	for(int i = 2; i <= k; i++)
	{
	int limit = sqrt(i);
	char is_prime = 1;
	for(int j = 2; j <= limit; j++)
	{
		if(i % j == 0)
		{
			is_prime = 0;
			break;
		}
	}
	if(is_prime)
		printf("%d,", i);
	}
	return 0;
}

\end{lstlisting}
\\
