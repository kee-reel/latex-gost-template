\chapter*{Задание 1. Вставка с сдвигом}

Вставить число 0 в середину массива М (20), предварительно сдвинув вправо значения элементов массива, начиная с 11-го. Выделить память сразу под 21 элемент. В особых случаях помещать число слева от предполагаемого "серединного числа".

\paragraph{Входные параметры}

\begin{tabular}{ |c|c|c|c| }
\hline
Название & Размер массива & Тип & Допустимые значения \\ 
 \hline
a & 21 & int & [-100, 100] \\ 
 \hline

\end{tabular}


\paragraph{Выходные параметры}

Вывести измененный массив с разделителем в виде вертикального слэша - пример 1|2|3 и т.д.
\\
\chapter*{Задание 2. Сортировка по убыванию}

Упорядочить линейный массив в порядке невозрастания значений его элементов.

\paragraph{Входные параметры}

\begin{tabular}{ |c|c|c|c| }
\hline
Название & Размер массива & Тип & Допустимые значения \\ 
 \hline
A & 6 & int & [-25, 25] \\ 
 \hline

\end{tabular}


\paragraph{Выходные параметры}

Вывести элементы результирующего массива. Выводить значения массива, разделяя их через вертикальный слеш -- пример: 1|2|3|4|5|
\\
\chapter*{Задание 3. Поменять первый и максимальный элемент}

В каждой строке матрицы А (7х9) поменять местами первый элемент и максимальный по модулю, если максимальное по модулю значение встречается в строке только один раз. При множественном повторении производить изменения не требуется. 

\paragraph{Входные параметры}

\begin{tabular}{ |c|c|c|c| }
\hline
Название & Размер массива & Тип & Допустимые значения \\ 
 \hline
A & 7x9 & int & [-100, 100] \\ 
 \hline

\end{tabular}


\paragraph{Выходные параметры}

Выводить значения матрицы, разделяя элементы строк и строки через вертикальный слеш -- пример для матрицы 2х2 (1 2)(3 4): 1|2||3|4||
\\
\chapter*{Задание 4. Сумма элементов матрицы}

Вычислить сумму элементов квадратной матрицы, лежащих справа от побочной диагонали.

\paragraph{Входные параметры}

\begin{tabular}{ |c|c|c|c| }
\hline
Название & Размер массива & Тип & Допустимые значения \\ 
 \hline
A & 10x10 & int & [-100, 100] \\ 
 \hline

\end{tabular}


\paragraph{Выходные параметры}

Вывести сумму элементов одним целым числом.
\\
\chapter*{Задание 5. Сессия}

Результаты сессии, состоящей из четырех экзаменов, для трех групп из 25 студентов представлены трехмерным массивом 3х25х4. Оценка ставится по четырехбалльной системе; неявка обозначена единицей. Определить, какая группа лучше подготовилась к сессии, получив более высокий средний балл.

\paragraph{Входные параметры}

\begin{tabular}{ |c|c|c|c| }
\hline
Название & Размер массива & Тип & Допустимые значения \\ 
 \hline
A & 3x25x4 & int & [1, 5] \\ 
 \hline

\end{tabular}


\paragraph{Выходные параметры}

Вывести номер группы (номера начинаются с 1) с самым высоким средним баллом и ее средний балл (с точностью до одного знака) через разделитель вертикальный слэш. Пример: 2|4.3
Если две и более групп имеют одинаковый средний балл, то вывести группу с наименьшим номером.
\\
